Als Jugendlicher fühlt er sich der Arbeit seiner Mutter sehr verbunden. Er unterstützt sie und die Schwarzseher bei den Aktionen zur Vertreibung von Grab- und Leichenräubern aus den Mórrsgärten und erlebt auch hautnah mit, wie eine kleine Gruppe laienhafter Nekromanten nachhaltig von ihrem fehlgeleiteten Weg abgebracht wird – selbstverständlich inklusive der anschließend notwendigen Mórrsriten.

Doch je öfter er die Schwarzseher begleitet und unterstützt, desto seltsamer fühlt sich alles für ihn an. Warum trauern Menschen, wenn Familienmitglieder die Schwarze Rose pflückten? Weshalb freuten sie sich nicht, dass diese die Reise in das Reich Mórrs antreten durften? Wieso wurden die Mórrspriester, je älter sie wurden und je länger sie ihrem Gott dienten, so morbide, melancholisch und menschenfremd? Wieso beschlossen Einige, sich selbst zum Tanze mit Mórr einzuladen? 

Sein Vertrauen in die Diener des Mórrkultes erstirbt in der Nacht, in der sie endlich einen Leichenräuber zur Strecke bringen, der ihnen mehrere Nächte hintereinander spurlos entwischt war. Tief sitzen Erschrecken und Erstaunen, als sich herausstellt, dass es sich um einen Hohepriester des Kultes handelt.

Im Alter von 14 Jahren beschließt Albrecht nun, Shallya an ihres Vaters statt die Ehre zu erweisen. 