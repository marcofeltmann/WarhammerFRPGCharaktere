Zum ersten Mal, seit er auszog um den Lehren seiner Göttin zu folgen, befindet Albrecht sich wieder im Middenland. Er kann bereits den Fauchslag sehen. Es ist nur noch ein kurzes Stück, maximal ein halber Tagesmarsch, bis er nach über acht Jahren endlich wieder Middenheim betritt.

Verändert hat er sich über die Jahre. Eine Größe von 180\,cm und eine schlanke bewegliche Figur hätte ihm damals kaum jemand zugetraut. Die Haare sind ein wenig strohfarbener geworden, doch seine nach wie vor purpurfarbenen Augen lassen jeden Zweifel schwinden. Der kleine, eher langsame und pummelige Albrecht hat sich zu einem attraktiven jungen Mann entwickelt. Der optische Eindruck wird lediglich durch die offensichtlich mehrfach gebrochene Nase ein wenig getrübt. Olfaktorisch ist die Attraktivität des jungen Mannes hingegen nicht so hoch angesiedelt, da man in seiner Gegenwart ständig den dezenten Geruch nasser, moosiger und halb verrotteter Erde mit einem Hauch von Raubtierexkrementen in der Nase hat.

Eine weiße Robe mit gold gesticktem Herzen an der linken Brust, die praktische und offenbar gut gefüllte Umhängetasche sowie der fast zwei Meter lange Wanderstab lassen keinen Zweifel daran, dass dieser Priester im Auftrag Shallyas wandert. 

\subsection*{Gastlichkeit}
Er hat seine Eltern besucht und auch ein paar Tage bei ihnen genächtigt. Leider war Selina, seine kleine Schwester, nicht dort gewesen. Die \enquote{Kleine} war ja mittlerweile auch schon 18 Jahre alt und befand sich in der akademischen Ausbildung zur Laienpriesterin Verenas. Sie kam seit je her eher nach ihrem Vater.

Albrecht ist es nicht mehr gewohnt, lange Zeit an einem Ort zu verweilen, weshalb er nach wenigen Tagen in den Ordenstempel in Middenheim einzieht.
Natürlich ist ihm die Gastlichkeit seines Ordens gewiss. Ebenso natürlich ist auch, dass er nach wenigen Tagen der Erholung den Tempel der Shallya wieder verlassen muss. 

Als er am örtlichen Waisenhaus vorbei kommt spendet er den Zehnt seiner Goldkronen. 
Er gerät ins Sinnieren. Sich die Entbehrlichkeit in seiner Jugend bewusst machend ist er froh und glücklich eine Familie zu haben. \enquote{Ein Waisenhaus leiten ist doch auch eine schöne Option für die Zukunft. In manchen Fällen angenehmer als ein Altenpflegehaus…} murmelt er sich selbst zu, wischt dann mit einer Geste seine Gedanken zur Seite und begibt sich zur nächsten Taverne.


Tavernen haben ihm immer gute Dienste geleistet, wenn es darum ging neue Ziele für seine Wanderpredigten ausfindig zu machen. Er besorgt sich beim Wirt ein wenig Proviant für die Reise nach Untergard, als kurz darauf eine zünftige Schlägerei zwischen mehreren Zwergen los bricht.
Schmunzelnd rettet sich Albrecht nach draußen, wartet den Tumult ab und geht dann wieder hinein, um ein paar leichte Schnittwunden zu versorgen.
Quasi nebenbei erfährt er, dass im Süden Middenlands das Chaos Einzug hält. Gerade soll Untergard irgendeinem Ansturm von zerstreuten Gruppen von Tiermenschen zum Opfer gefallen sein.
Damit steht sein nächstes Reiseziel fest: Middenlands Süden, gegebenenfalls Untergard. Dummerweise dämmert es bereits, so dass ein sofortiger Aufbruch nicht so klug ist.

Er verbringt also die letzte Nacht im Waisenhaus. Hier verkündet Albrecht den Kindern das Werk Shallyas, beantwortet bereitwillig Fragen bezüglich des legendären Familienzwists zwischen ihr und ihrem Vater, hilft bei der Essensausgabe und geleitet die Kleinen ins Bett, bevor er sich auf seinem Reiselager auf dem Küchenboden zum Schlafen einrollt.

Der folgende Tag steht ganz im Zeichen des Aufbruchs.
